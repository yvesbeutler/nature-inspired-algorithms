\chapter{Code-Listings}

\section{Genetischer Algorithmus}

\begin{lstlisting}[language=Python, caption=Genetischer Algorithmus (Python)]
    class Individual:
      def __init__(self, genes):
          self.genes = genes
          self.fitness_function(genes)
      
      # calculates fitness of each individual
      def fitness_function(self, genes):
          score = 0
          for gene in genes:
              if gene == 1:
                  score += 1
          self.fitness = score
    
    class Population:
        def __init__(self, individuals):
            self.individuals = individuals
        
        def crossover(self, parent_1, parent_2, cross_value=3):
            offspring_1 = parent_1
            offspring_2 = parent_2
            
            # exchange genes below crossover point
            for i in range(cross_value-1):
                gen = offspring_1.genes[i]
                offspring_1.genes[i] = offspring_2.genes[i]
                offspring_2.genes[i] = gen
            
            return [offspring_1, offspring_2]
    
    def genetic_algorithm():
        generation = 1
        population = Population([
            Individual([0, 1, 0, 1, 0]), Individual([1, 1, 0, 0, 0]),
            Individual([0, 1, 1, 1, 0]), Individual([0, 0, 0, 1, 1]),
            Individual([1, 0, 0, 0, 1]), Individual([1, 0, 1, 1, 0])])
    
        while population.get_fittest().fitness < 5:
            # selection
            parent_1 = population.get_fittest()
            parent_2 = population.get_fittest()
    
            # crossover
            offspring = population.crossover(parent_1, parent_2)
    
            # mutation
            offspring = population.mutate(offspring)
    
            # add offspring, remove weakest individuals
            population.grow(offspring)
            population.kill_weakest()
            generation += 1
    
        print(f'The population converged after {generation} generations')
    \end{lstlisting}

\clearpage
\newpage
