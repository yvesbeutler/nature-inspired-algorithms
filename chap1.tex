%% This is an example first chapter.  You should put chapter/appendix that you
%% write into a separate file, and add a line \include{yourfilename} to
%% main.tex, where `yourfilename.tex' is the name of the chapter/appendix file.
%% You can process specific files by typing their names in at the 
%% \files=
%% prompt when you run the file main.tex through LaTeX.
\chapter{Einführung}

Seit dem Anbeginn unseres Planeten sah sich die Natur mit mehr oder weniger
komplexen Problemstellungen konfrontiert, um das Überleben ihrer Bewohner
sicherzustellen. Der Schlüssel zum Erfolg ist es, sich am besten auf die
naturgegebenen Umstände anzupassen. Ohne den Einklang mit der Natur wären die
Überlebenschancen einer jeden Spezies schwindend gering wenn nicht gar inexistent.
Seit Jahrzehnten adaptieren Menschen die Techniken aus der Natur und ihren Geschöpfen
um moderne Technologien auf ein neues Level zu befördern. Flugzeuge werden nach den
Flügelcharakteristiken von Zugvögeln konstruiert und neue Klebstoffe werden durch das
Studium von Geckofüssen mit ihren unglaublichen Hafteigenschaften entworfen. \cite{Cro14}

Wir dürfen nicht ausser Acht lassen, dass die Natur Millionen von Jahre in die Forschung
nach dem Schlüssel zum Überleben investierte und in den meisten Fällen scheiterte. Doch
die Natur gab nicht einfach auf, sondern justierte bestimmte Parameter um über die Zeit
zunehmend besser zu werden. Durch sich ständig ändernde Umstände sucht sie auch heute noch
nach Lösungen für zukünftige Spezien.

Wieso sollte die Menschheit nicht selbst von diesen Millionen von Jahren an Forschung profitieren?

\section{Anwendungsgebiete}

Nature-Inspired Algorithms werden meist dort eingesetzt, wo traditionelle Algorithmen keine oder
keine brauchbaren Ergebnisse liefern können. Häufig scheitern altbekannte Lösungsmodelle wenn mit
riesigen Datenmengen gearbeitet wird. Folgende Gegebenheiten können weitere Gründe sein, um auf
Nature-Inspired Algorithms zurückzugreifen:

\begin{enumerate}
    \item Die Anzahl möglicher Lösungen im Suchraum ist so immens gross, dass die Suche nach der bestmöglichen
          Lösung viel zu aufwändig wäre.
    \item Die Komplexität der Problemstellung ist derart hoch, dass nur vereinfachte Modelle bei der Problemlösung
          zum Einsatz kommen und dadurch keine aussagekräftigen Lösungen entstehen.
    \item Die Evaluierungsfunktion zur Bewertung einer möglichen Lösung variert mit der Zeit, so dass nicht nur eine
          sondern eine Vielzahl von Lösungen erforderlich ist.
\end{enumerate}

\cite[Kap. 1.2]{Bro11} \\

Eine weitverbreitete Eigenschaft von Nature-Inspired Algorithms ist die Kombination aus Regeln und Zufälligkeiten, um
natürliche Phänomene zu imitieren \cite{NH15}. Die Einsatzmöglichkeiten scheinen unendlich zu sein. Häufig werden
solche Algorithmen dort eingesetzt, wo für die Problemstellung zwar bereits Lösungen existieren, diese aber noch viel
Verbesserungspotential hinsichtlich Zeitdauer, Ressourcenverbrauch oder Genauigkeit aufweisen.

Konkrete Anwendungsgebiete werden bei den entsprechenden Ausprägungen der einzelnen Algorithmen im Detail behandelt.


\subsection{Optimierungen}
Häufige Anwendungsfälle dieser Algorithmen sind Optimierungen von Funktionen. Bei Optimierungen
handelt es sich in der Regel um die Suche nach einer Parameterkombination für eine gegebene Funktion $f$ um eine
Kostenfunktion zu minimieren oder eine Wertefunktion zu maximieren.

\subsection{Approximationen}
Die Approximation beschreibt meist eine Funktion $f$, welche möglichst nahe an eine Zielfunktion angenähert werden
möchte. Die approximierte Funktion $f$ wird aus einem Set an Beobachtungen\footnote{oftmals auch als $training$ $set$ aus
der Data Science bekannt} generiert. Solche Approximationen werden häufig für Image Recognition verwendet
und spielen ebenfalls eine wichtige Rolle bei der Klassifikation und dem Clustering von grossen Datenmengen.